\documentclass{10extra/workingpaper}

\usepackage{indentfirst}

\addbibresource{09literature/bibliography.bib}

\title{Criminal Mines: Evaluating the effects of the coal mining industry on incarceration and drug-related crime\\[0.5em]{\Large Introduction and Literature Review}}
\author[*]{Robert Betancourt}
\author[*]{Connor Bulgrin}
\author[$\dagger$]{Jenny Duan}
\author[*]{Nicholas Skelley}
\author[*]{Adeline Sutton}

\affil[*]{\normalsize Department of Economics, Cornell University}
\affil[$\dagger$]{\normalsize Jeb E. Brooks School of Public Policy, Cornell University}
% \keywods
% \jel{K14, L71, O13, O14, O33}

\date{Last updated: \today}

\begin{document}

\maketitle
% \maketitlepage


\setstretch{1.5}

\section{Introduction}
\subsection{Background}

The past century has seen a sharp decline in the amount of coal mined and consumed. \tk{[INSERT SENTENCE ABOUT COAL MINED 1900-2000]} The past two decades alone have seen a large drop due to decreased demand from the power sector. This was primarily caused by technological advances, which were associated with increased natural gas extraction, especially fracking. Between 2001 and 2023, aggregate coal productivity in the US has gone from 1.13 billion short tons to 578 million, a 48.75\% decrease \parencite{coal_data_browser_eia}. In the same period, natural gas growth withdrawals have increased from 24.5 million (million cubic feet) to 45.6 million \parencite{u.s.energyinformation_2025}.
% https://www.eia.gov/coal/data/browser/#/topic/33?agg=2,0,1&geo=nvg1qag9vvlpns&linechart=COAL.PRODUCTION.TOT-US-TOT.A&columnchart=COAL.PRODUCTION.TOT-US-TOT.A&map=COAL.PRODUCTION.TOT-US-TOT.A&freq=A&start=2000&end=2023&ctype=linechart&ltype=pin&rtype=b&pin=&rse=0&maptype=0&rank=g&mntp=g
%https://www.eia.gov/dnav/ng/hist/n9010us2a.htm

In Appalachia, coal is king. In 2005, the coal industry provided an average wage in the top third of sectors nationwide \parencite{itkin_2006}. High wages, along with the opportunity for work, offers a lucrative opportunity despite the dangers associated with mine labor. \textcite{young_2023} reported that one interviewee described the importance of coal as follows, ``When things are good in the mine, things are good everywhere in the county.” Beginning in 2001, close to the start of the coal industry's decline, outmigration surged. Figure \ref{fig:change_in_pop} shows change in population during our period of interest. During the period from 2008 to 2019, the total income per county decreased from \tk{[INSERT] to [INSERT]}.

\begin{figure}
    \centering
    \includegraphics[width=0.5\linewidth]{08manuscript/placeholder.png}
    \caption{Source: \parencite{bowen_2020} \tk{BUT WE SHOULD MAKE SOMETHING SIMILAR OURSELVES}}
    \label{fig:change_in_pop}
\end{figure}

Prior research has explored the effect of the coal decline on social effects. \textcite{blonz_2023} find decreased credit scores and increased use of credit, even when accounting for individuals who lost mining-related jobs. \textcite{welch_2020} study the impact of coal activity on local revenue, finding a decrease in local revenue for education per student. Naturally, we expect more poorly funded counties to have worse justice outcomes. A minority of the literature finds the opposite conclusion---for example, \textcite{nason_2025} find some evidence that long-term local reduction in poverty was due to outmigration, wage increases and increased educational investment. In order to determine which counties are the most vulnerable, \textcite{raimi_2022} create county-level measures for socioeconomic vulnerability to energy transitions. We repeat their same method to create pre-period measures to use as a robustness check for our main results.

A large portion of the literature has focused on opioid addiction in Appalachia. \textcite{young_2023} and \textcite{metcalf_2019} disagree on whether the decline of the coal industry led to an increase in opioid use. One possible mechanism is that decreasing mining activity resulted in more despair, increasing opioid use, while the other is that it resulted in the elimination of mining jobs whose injuries led to high opioid use. We expand on the above literature by paying close attention to drug-related crime and opioid mortality rates to better understand whether the decline of the coal industry contributed to the opioid crisis in Appalachia.

This paper ties into the literature on whether poor economic conditions are causally linked to worse crime outcomes. School quality has been well documented to be associated with a reduction in crime (\cite{deming_2011}; \cite{hjalmarsson_2015}; \cite{meghir_2011}; \cite{bennett_2018}; \cite{lochner_2004}). Our paper explores the mechanism by which deindustrialization contributes to an employment shock and outmigration, which then leads to lower local revenues, and hence, decreased educational funding.

Some literature has also examined the relationship between deindustrialization, health outcomes, and crime statistics. \textcite{nosrati_2018} found that between 2001 and 2014 in the U.S., deindustrialization and incarcaration reduced the lifespan of the poor by around 2.5 years. \tk{[INSERT SOME MORE OF THAT LITERATURE HERE]} Arguably the most similar paper to ours is \textcite{white_1999} who examined deindustrialization between 1970 and 1990 in the 100 largest US cities, finding inconsistencies in correlations between the combination of deindustrialization and increased poverty with increases in various crime measures.

Our broad question of interest is ``How does deindustrialization impact crime outcomes?'' More specifically, we test whether a decline in a county's share belonging to coal causes an increase in various crime statistics, including violent, property, and drug-related crime.

\subsection{De-industrialization and Exposure Shock}
Most other papers that look at declines in industrialization or manufacturing that use exposure shocks use exposure as a continuous variable. \cite{howard2024universities} and \cite{glaeser2003rise} both use baseline manufacturing employment shares (the peak of U.S manufacturing employment) as a measure for manufacturing exposure in different years. Similarly \cite{gagliardi2023rust} looks at the effect of de-industrialization on employment consequences in cities across six countries and uses the city's share of manufacturing employment in the year of its country’s manufacturing peak as treatment for subsequent changes in total employment.  Former manufacturing hubs are defined as cities with an initial manufacturing employment share in the top tercile of their country's distribution while middle and bottom terciles are used for comparison in their 2SLS regression using driving distance to historical colleges and universities as their IV.  \cite{autor2013china} analyze the effects of rising Chinese import competition between 1990 and 2007 on US local labor markets, exploiting cross-market variation in import exposure stemming from initial differences in industry specialization. They define exposure as local industry employment shares in time t and compare percentiles (25th and 75th percentiles) as well as the median to contextualize magnitudes of exposure. 


\subsection{Data}

For our purposes, we define deindustrialization as ``the decrease in share of a county's industry belonging to coal."

We measure the share of county employment accounted for by the coal industry using the Quarterly Census of Wages and Employment (QCEW) data from the U.S. Bureau of Labor Statistics (BLS). QCEW data are available at the county-year-industry-ownership level, where ownership distinguishes between public (government) and private ownership, and industries are available as NAICS codes down to the ``national industry'' \citep{naics_2022} level.

We use data for NAICS code 10, defined in the QCEW as representing all industries, as the data representing overall wages and employment in the county, and data for NAICS code 2121 for coal-industry employment. NAICS code 2121 is defined as ``coal mining'' and includes mining of all subtypes of coal---including bituminous, lignite, and anthracite---in both underground and aboveground mines.

Regarding crime, we use data published by the Justice Outcomes Explorer (JOE), which combines the Criminal Justice Administrative Records System (CJARS) with socioeconomic data from the U.S. Census Bureau \tk{CITE THIS}. The JOE features extensive crime statistics at the county level. In certain cases, particularly when small population counts mean that county-level data would break anonymity, the JOE does not provide values. \tk{INSERT WHAT PORTION OF COUNTIES IN APPALACHIA WE HAVE DATA FOR}

%------------------------------------------------------------------------------%

\newpage
%\bibliographystyle{apacite}
%\nocite{*}
%\bibliography{09literature/bibliography.bib}
\printbibliography

\end{document}

%------------------------------------------------------------------------------%

10/28/2025 Notes
Problem: perfect overlap in coal and gas counties production
jobs aren't one-to-one coal to nat gas
there's a transition cost
if outcomes jump up and back down maybe that's the 
no longer addicted cuz now there's a job for me! <--- NO!
safety of coal mines
nationwide gas for robustness

Now we are doing "classic" Bartik
