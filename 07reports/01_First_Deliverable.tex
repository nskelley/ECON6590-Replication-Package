\documentclass{10extra/creport}
\usepackage{rotating}
% Hacky way of removing the month from the bibliography dates
\AtEveryBibitem{\clearfield{labelmonth}}

\addbibresource{09literature/bibliography.bib}

\title{Criminal Mines: Evaluating the effects of the coal mining industry on incarceration and drug-related crime}
\author{Robert Betancourt \qquad Connor Bulgrin \qquad Jenny Duan \\[0.5em] Nicholas Skelley \qquad Adeline Sutton}

\renewcommand{\cite}{\textcite}

\date{October 2025}

\begin{document}
\maketitle

\setstretch{1.5}

\section{Introduction}
Our broad question of interest is, ``How does deindustrialization impact crime outcomes?''  Specifically, we plan to examine the collapse of the coal industry in Appalachia, where coal was a significant industry up until the rise of natural gas in the United States in the early 2000s.  

Prior research has explored the effect of the coal decline on social and economic outcomes. \cite{blonz_2023} find decreased credit scores and increased use of credit, even when accounting for individuals who lost mining-related jobs. \textcite{welch_2020} study the impact of coal activity on local revenue, finding a decrease in local revenue for education per student. Naturally, we expect more poorly funded counties to have worse justice outcomes. A minority of the literature finds the opposite conclusion---for example, \textcite{nason_2025} find some evidence that long-term local reduction in poverty was due to outmigration, wage increases and increased educational investment. Similarly, \textcite{young_2023} and \textcite{metcalf_2019} disagree on whether the decline of the coal industry led to an increase in opioid use. One possibility is that widespread job loss and the revenue losses that ripple outward from it lead to more despair and, therefore, increased opioid use. Alternatively, the elimination of mining jobs may reduce the risk of injuries and long-term consequences arising from the dangers and environmental hazards faced in mines, decreasing the likelihood of being in a position to be prescribed an opioid and the prevalence of opioids throughout the community. With these considerations in mind, we pay close attention to drug-related justice outcomes.

\section{Design and Data}

\subsection{Plan and Summary}

We plan to investigate the relationship of deindustrialization via the decline of the coal industry in Appalachia. Because deindustrialization is not exogenous, we instrument for it with a shift-share instrument \citep{bartik_1991} that measures the exposure of a county to changes in the coal industry. Counties whose labor markets depend more heavily on coal are more vulnerable to fluctuations and permanent changes in the industry.

\subsubsection{Natural gas-coal threat to identification}

A threat to our identification strategy is that counties with large coal-mining sectors often transition to natural gas production from coal, and many natural-gas-producing counties were previously dependent on coal mining. We argue that while counties may substitute \emph{production} of one type of energy for the other, \emph{jobs} are not easily substitutable; coal mining and natural gas production require different skills and complementary capital investments, and they pose different risks to workers. Furthermore, search frictions in the labor market imply that workers who lose their jobs as coal miners will experience a period of unemployment between jobs, even if they would be optimal matches for natural gas producers, and they may never successfully match at all (e.g., by dropping out of the labor force).

A possible mechanism for our proposed effect of de-industrialization on crime is increased drug---especially opioid---addiction among the unemployed and those who develop chronic conditions (specifically, chronic pain or respiratory illnesses) through workplace hazards. Workers moving from coal mining to natural gas production will not be subject to the same hazards at work (e.g., coal dust, risk of mine collapse, reduced exposure to light and sunlight) after transitioning. However, they will also not ``reverse'' an addiction developed during the period of unemployment once offered a job---while getting a new job may be motivation to ``get clean,'' overcoming addiction is a considerable undertaking requiring an extended period of commitment and support.

\subsubsection{Testing for substitution between natural gas and coal employment}

To test for coal and natural gas jobs being perfect (one-to-one) and relatively frictionless substitutes, we can evaluate the relationship between coal and natural gas employment within each county-year grouping. If we perform the OLS regression
\[
    C_{st} = \alpha + \beta \, G_{st} + \varepsilon_{st},
\]
where $s$ denotes the county, $t$ denotes the year, $C_{st}$ denotes the \emph{number} of coal mining jobs within state $s$ in year $t$, and $G_{st}$ denotes the number of natural gas jobs in $s$ and $t$, then under one-to-one, frictionless substitution, we would expect $\beta = -1$.

% \tk{In the above, should change to state (to account for possible shifts in employment from one county to a neighboring one as coal jobs transition to natural gas); also need location (state) and time(?) fixed effects.} \nb{Quick first pass at this regression actually showed a precise $\beta > 0$ not statistically distinguishable from $0$.}

\subsubsection{Testing relative dangers}

One method of quantifying the relative dangers present in coal mining work compared to natural gas production is to examine federal- and state-level regulations on each industry (e.g., pages of OSHA regulations that apply specifically to each sector). Alternatively, we could examine workplace injury and mortality statistics or worker compensation claims within each industry. There are also unobserved hazards associated with each type of energy production that rarely manifest in a salient way at the workplace---chief among them, cancer and illnesses caused by exposure to harmful ambient gases (methane, benzyne, and other cancer- and respiratory-illness-causing volatile organic compounds) and particulate pollutants (coal dust).

\subsection{Data}

\subsubsection{Employment share}

We measure the share of county employment accounted for by the coal industry using the Quarterly Census of Wages and Employment (QCEW) data from the U.S. Bureau of Labor Statistics (BLS). QCEW data are available at the county-year-industry-ownership level, where ownership distinguishes between public (government) and private ownership, and industries are available as NAICS codes down to the ``national industry'' \citep{naics_2022} level.

We use data for NAICS code 10, defined in the QCEW as representing all industries, as the data representing overall wages and employment in the county, and data for NAICS code 2121 for coal-industry employment. NAICS code 2121 is defined as ``coal mining'' and includes mining of all subtypes of coal---including bituminous, lignite, and anthracite---in both underground and aboveground mines.

\subsection{Design and specification}

\subsubsection{Theoretical Framework}

For county $k$ and year $t$, we propose the 2SLS framework specified below.

\begin{align}
    Y_{kt} &= D_{kt}\beta + X_{kt}\gamma + \kappa_k + \tau_t + \varepsilon_{kt} \tag{1} \\
    Z_{kt} &= \texttt{coal\_share}_{k,2002} \cdot \texttt{coal\_growth}_t \tag{2} \\
    D_{kt} &= Z_{kt}\phi + X_{kt}\tilde{\gamma} + \tilde{\kappa}_k + \tilde{\tau}_t + \tilde{\varepsilon}_{kt} \tag{3} \\
    Y_{kt} &= \hat{D}_{kt}\beta + X_{kt}\gamma + \kappa_k + \tau_t + \varepsilon_{kt} \tag{4}
\end{align}

In the above, $Y$ denotes the outcome of interest (e.g., misdemeanors per capita), $D$ is the measure of deindustrialization (the change in coal employment), $X$ is a vector of county-level control variables, 
$\kappa_k$ are county fixed effects, and $\tau_t$ are year fixed effects. 
The instrument is a \citeauthor{bartik_1991}-style (\citeyear{bartik_1991}) instrument defined as $Z_{kt} = \texttt{coal\_share}_{k,2002} \cdot \texttt{coal\_growth}_t$, where $\texttt{coal\_share}_{k,2002}$ is county $k$'s baseline dependence on coal employment (measured as the share of total employment in coal mining in the year 2002), and  $\texttt{coal\_growth}_t$ is the national growth rate of coal employment in year $t$, normalized so that $\sum_k \texttt{coal\_share}_{k,2002} = 1$.

\subsubsection{Identification assumptions}

\begin{enumerate}
    \item \textbf{Relevance:} $\text{Cov}(Z_{kt}, D_{kt} \mid X_{kt}, \kappa_k, \tau_t) \neq 0.$ The instrument (coal dependence × national coal growth) actually shifts local coal employment.  It makes sense that there is a correlation between exposure to coal shocks and the decline of the coal sector in a county because the more dependent a county is on coal, the more affected the local economy will be by coal shocks.  %\tk{We test for this in the data and summary statistics section.}
    \item \textbf{Exogeneity of shocks:} $\texttt{coal\_growth}_t$ is uncorrelated with county-level shocks to $\varepsilon_{kt}.$ I.e., national coal trends aren’t driven by county-level crime or local conditions. This is a reasonable assumption to make since a county is so small relative to the country that it is unlikely for crime conditions in the county to cause a national shock.  %we can check for outliers in the data.  If there is low variance in crime conditions across counties and high variance in coal production, it's unlikely crime conditions in a county are influencing coal prod
    \item \textbf{Predetermined exposure:} $\text{coal\_share}_{k,2002}$ is measured prior to the study period and not affected by future outcomes.
    \item \textbf{Exclusion restriction:} $Z_{kt}$ affects $Y_{kt}$ only through $D_{kt}$. The Bartik instrument affects crime outcomes only through changes in coal employment, and not through other channels.
    \item \textbf{Stable unit treatment value assumption (SUTVA):} Outcomes in county $k$ are unaffected by coal shocks in other counties. We have reason to believe that such spillovers are likely, as coal job losses and gains may occur in one county, but workers may live nearby instead. To explore this possibility, we will conduct analyses on counties bordering non-zero-coal-job counties.
    \item \textbf{Monotonicity:} Higher $\texttt{coal\_growth}_t$ weakly increases $D_{kt}$ for all $k$. When national coal jobs rise, every coal-dependent county moves in the same direction (no “defiers”). \cite{dechaisemartin_2017} claims that instrumental variables are still valid when ``one can reasonably assume that defiers' LATE has the same sign as the reduced form effect of the instrument on the outcome, or that compliers’ and defiers’ LATEs do not differ too much''.  In our context, a defier would be a county that, after experiencing a negative shock to coal, increased industry.  For example, they might rapidly transition to natural gas production.  However, in most cases, it is reasonable to assume that there is a transition cost and lag between coal firm exit and new firm entry, allowing the LATEs to be the same in the short run.  %\nb{This may not hold, but I wonder if the Dechaisemartin (2017) (more compliers than defiers at all times) assumption can help bail us out!}
\end{enumerate}


\section{Summary Statistics}

Figure \ref{fig:linechart} shows average crime rates over time for Appalachian states of interest, overlaid with the percent change in coal jobs during the same period. We see that crime statistics in the area appear largely unaffected by the dramatic decline in the coal industry in 2015. Figures \ref{fig:binscatter_felony} and \ref{fig:binscatter_misdemeanor} present more robust statistical evidence in the style of \cite{cattaneo_2024} and offer suggestive evidence that counties experiencing a significant drop in coal employment may see slightly higher crime outcomes. 

\begin{figure}[htbp]
    \centering
    \includegraphics[width=0.8\textwidth]{04work/early_figures/line_chart_offense.pdf}
    \caption{Felony Rates Over Time by Offense Type}
    \label{fig:linechart}
\end{figure}

\begin{figure}[!htbp]
    \centering
    \includegraphics[width=0.8\textwidth]{04work/early_figures/binscatter_fe_rate.pdf}
    \caption{Binned Scatterplot of Felony Rates by Changes in Coal Employment}\vspace{0.5em}
    \label{fig:binscatter_felony}
    
    \begin{minipage}{0.9\textwidth}
        \small
        \textit{Note:} Felony offenses per 100,000 residents, by county, plotted against percent changes in the share of jobs accounted for by coal mining within each county. Observations are weighted by the county's average total employment during the year of observation and include controls for sex, race, age group, county, the share of jobs accounted for by coal mining in 2002, and the year of observation. Quantile-based bins using \texttt{binsreg} \citep{binsreg}, following \cite{cattaneo_2024}.
    \end{minipage}
\end{figure}

\begin{figure}[!htbp]
    \centering
    \includegraphics[width=0.8\textwidth]{04work/early_figures/binscatter_mi_rate.pdf}
    \caption{Binned Scatterplot of Misdemeanor Rates by Changes in Coal Employment}\vspace{0.5em}
    \label{fig:binscatter_misdemeanor}
    
    \begin{minipage}{0.9\textwidth}
        \small
        \textit{Note:} Misdemeanor offenses per 100,000 residents, by county, plotted against percent changes in the share of jobs accounted for by coal mining within each county. As in Figure \ref{fig:binscatter_felony}, observations include weights from average total employment and controls for demographics, time, and share of jobs accounted for by coal mining in 2002 and are summarized according to the procedure described by \cite{cattaneo_2024}.
    \end{minipage}
\end{figure}

Table \ref{tab:summary_offense} shows simple means for per-capita crime and incarceration rates in Appalachia during our time period of interest, 2006--2019. Table \ref{tab:summary_multivar} identifies summary statistics for the Appalachian counties we are most interested in and the significance of coal mining in their local economies. While many counties do not have any coal jobs, those that do have up to almost 30\% of their jobs accounted for by coal mining.

\begin{table}[htbp]
    \centering
    \small
    \caption{Summary Statistics by Offense Type}
    \label{tab:summary_offense}
    \begin{tabular}{@{\extracolsep{5pt}} l c c c c} 
        \\[-1.8ex]\hline 
        \hline \\[-1.8ex] 
        Offense Category & N & \parbox{2.5cm}{\centering Felony Rate} & \parbox{2.5cm}{\centering Incarceration Rate} & \parbox{2.5cm}{\centering Misdemeanor Rate} \\ 
        \hline \\[-1.8ex] 
        All Offenses & 66,822 & 1,080.11 & 365.86 & 4,332.17 \\ 
        Drug & 29,304 & 286.64 &  & 858.64 \\ 
        Property & 29,304 & 435.12 &  & 763.14 \\ 
        Violent & 29,304 & 217.72 &  & 675.62 \\ 
        \hline\hline \\[-1.8ex] 
    \end{tabular}
    \begin{minipage}{0.8\textwidth}
        \textit{Note:} Summary statistics represent unweighted means over counties and years of interest, separated by offense type. Felony, incarceration, and misdemeanor rates denote the rate of each type of encounter within each county per 100,000 county residents.
    \end{minipage}
\end{table}

\begin{table}[!htbp] \centering 
  \caption{Coal Employment Summary Statistics} 
  \label{tab:summary_multivar} 
    \begin{tabular}{@{\extracolsep{5pt}} lc} 
        \\[-1.8ex]\hline 
        \hline \\[-1.8ex] 
        Variable & Value \\ 
        \hline \\[-1.8ex] 
        Unique Counties & 53 \\ 
        Mean Coal Emp Share & 0.021 \\ 
        SD Coal Emp Share & 0.059 \\
        Min Coal Emp Share & 0 \\ 
        Max Coal Emp Share & 0.295 \\ 
        Mean Nat Coal Jobs \% Chg & -1.999 \\ 
        SD Nat Coal Jobs \% Chg & 15.165 \\\hline
        Year Range & 2006 -- 2019 \\
        Observations (County-Years) & 617 \\ 
        \hline\hline \\[-1.8ex] 
    \end{tabular} 
\end{table}

We run simple least-squares regressions of crime outcomes on changes in coal employment shares and present the results in Tables \ref{tab:felony_naive} and \ref{tab:misdemeanor_naive}. These OLS outcomes show a descriptive connection between changes in the coal mining industry and crime rates. However, we cannot interpret the results of a simple regression as causal, even when controlling for demographic factors, time, and idiosyncratic variation between counties. Deindustrialization (as measured by a drop in coal employment) and crime rates are most likely correlated with other factors affecting both variables, such as access to other areas, out-of-market unemployment, and unobservable characteristics endogenous to the choice to live and work in coal-dependent areas. That is to say, our error term will be correlated with our deindustrialization treatment variable, creating unobservable bias in our estimator. Other potential unobservables that could bias our estimator are pre-existing economic and social conditions; counties that are deindustrializing may also be more prone to having poor infrastructure or declines in other local industries indicative of a broader economic decline in the area. Demographic patterns for certain areas might also affect crime rates. For example, people who choose to migrate out of areas with fewer economic opportunities might be people with lower tendencies to commit crimes. Educational attainment is also a predictor of crime, and counties with populations that have lower educational attainment are more vulnerable to deindustrialization, as low-skill jobs are easier to replace, and may face faster rates of job loss than neighboring counties. Other sources of bias might be government policies or social programs intended to help people who are experiencing job loss---for example, job training programs, Medicaid expansions, and social safety nets.

For these reasons, we should not interpret the regression estimates presented in Tables \ref{tab:felony_naive} and \ref{tab:misdemeanor_naive} as causal. Our \citeauthor{bartik_1991}-style (\citeyear{bartik_1991}) instrumental-variable approach is designed to elicit a causal interpretation in order to better understand whether de-industrialization \textit{causes} crime in a county to increase.

\begin{sidewaystable}[!htp]
    \centering
    \caption{Na\"ive regressions of felony rates}
    \label{tab:felony_naive}
    
% Table created by stargazer v.5.2.3 by Marek Hlavac, Social Policy Institute. E-mail: marek.hlavac at gmail.com
% Date and time: Fri, Oct 31, 2025 - 16:40:50
\begin{table}[!htbp] \centering 
  \caption{} 
  \label{} 
\begin{tabular}{@{\extracolsep{5pt}}lcccccccccc} 
\\[-1.8ex]\hline 
\hline \\[-1.8ex] 
 & \multicolumn{10}{c}{\textit{Dependent variable:}} \\ 
\cline{2-11} 
\\[-1.8ex] & \multicolumn{10}{c}{fe\_rate} \\ 
\\[-1.8ex] & (1) & (2) & (3) & (4) & (5) & (6) & (7) & (8) & (9) & (10)\\ 
\hline \\[-1.8ex] 
 oty\_annual\_avg\_emplvl\_pct\_chg\_2121 & 0.580 & 0.583 & 1.254 & 1.261 & 0.612$^{**}$ & 0.614$^{**}$ & 0.514 & 0.517 & $-$0.061 & $-$0.059 \\ 
  & (0.379) & (0.370) & (1.319) & (1.268) & (0.296) & (0.287) & (0.492) & (0.470) & (0.445) & (0.434) \\ 
  & & & & & & & & & & \\ 
 factor(off\_type)1 & $-$862.390$^{***}$ & $-$862.390$^{***}$ &  &  &  &  &  &  &  &  \\ 
  & (8.648) & (8.430) &  &  &  &  &  &  &  &  \\ 
  & & & & & & & & & & \\ 
 factor(off\_type)2 & $-$644.986$^{***}$ & $-$644.986$^{***}$ &  &  &  &  &  &  &  &  \\ 
  & (8.648) & (8.430) &  &  &  &  &  &  &  &  \\ 
  & & & & & & & & & & \\ 
 factor(off\_type)3 & $-$793.474$^{***}$ & $-$793.474$^{***}$ &  &  &  &  &  &  &  &  \\ 
  & (8.648) & (8.430) &  &  &  &  &  &  &  &  \\ 
  & & & & & & & & & & \\ 
\hline \\[-1.8ex] 
Demographic controls & No & Yes & No & Yes & No & Yes & No & Yes & No & Yes \\ 
Observations & 111,840 & 111,840 & 27,960 & 27,960 & 27,960 & 27,960 & 27,960 & 27,960 & 27,960 & 27,960 \\ 
R$^{2}$ & 0.123 & 0.167 & 0.041 & 0.113 & 0.069 & 0.125 & 0.059 & 0.141 & 0.028 & 0.076 \\ 
Adjusted R$^{2}$ & 0.123 & 0.166 & 0.039 & 0.112 & 0.067 & 0.124 & 0.057 & 0.140 & 0.027 & 0.075 \\ 
Residual Std. Error & 1,022.473 (df = 111800) & 996.729 (df = 111797) & 1,778.400 (df = 27923) & 1,709.749 (df = 27920) & 399.253 (df = 27923) & 386.936 (df = 27920) & 663.323 (df = 27923) & 633.541 (df = 27920) & 600.377 (df = 27923) & 585.436 (df = 27920) \\ 
F Statistic & 401.879$^{***}$ (df = 39; 111800) & 532.053$^{***}$ (df = 42; 111797) & 32.867$^{***}$ (df = 36; 27923) & 91.552$^{***}$ (df = 39; 27920) & 57.047$^{***}$ (df = 36; 27923) & 102.447$^{***}$ (df = 39; 27920) & 48.285$^{***}$ (df = 36; 27923) & 117.832$^{***}$ (df = 39; 27920) & 22.609$^{***}$ (df = 36; 27923) & 59.038$^{***}$ (df = 39; 27920) \\ 
\hline 
\hline \\[-1.8ex] 
\textit{Note:}  & \multicolumn{10}{r}{$^{*}$p$<$0.1; $^{**}$p$<$0.05; $^{***}$p$<$0.01} \\ 
\end{tabular} 
\end{table} 

    \begin{minipage}{0.96\textwidth}
        \small\vspace{-1.5em}
        \textit{Notes:} All regressions include county and year fixed effects. In ``Felony rate'' columns, the excluded group is the total/all offenses type. \\
        $^* p<0.1$ \, $^{**} p<0.05$ \, $^{***} p<0.01$
    \end{minipage}
\end{sidewaystable}

\begin{sidewaystable}[!htp]
    \centering
    \caption{Na\"ive regressions of misdemeanor rates}
    \label{tab:misdemeanor_naive}
    
\begin{tabular}{@{\extracolsep{0pt}}lcccccccccc} 
\\[-1.8ex]\hline 
\hline \\[-1.8ex] 
 & \multicolumn{2}{c}{Misdemeanor rate} & \multicolumn{2}{c}{All misdemeanors} & \multicolumn{2}{c}{Violent misdemeanors} & \multicolumn{2}{c}{Property misdemeanors} & \multicolumn{2}{c}{Drug misdemeanors} \\ 
 \begin{minipage}{30mm}
	\raggedright
	\% change in coal employment
\end{minipage} & 1.587 & 1.596 & 4.927 & 4.950 & 1.545$^{**}$ & 1.548$^{**}$ & 0.216 & 0.220 & $-$0.339 & $-$0.334 \\ 
  & (1.177) & (1.154) & (4.403) & (4.241) & (0.783) & (0.758) & (0.797) & (0.764) & (1.043) & (1.007) \\ 
  & & & & & & & & & & \\ 
 Violent offense & $-$3,656.551$^{***}$ & $-$3,656.551$^{***}$ &  &  &  &  &  &  &  &  \\ 
  & (27.152) & (26.620) &  &  &  &  &  &  &  &  \\ 
  & & & & & & & & & & \\ 
 Property offense & $-$3,569.030$^{***}$ & $-$3,569.030$^{***}$ &  &  &  &  &  &  &  &  \\ 
  & (27.152) & (26.620) &  &  &  &  &  &  &  &  \\ 
  & & & & & & & & & & \\ 
 Drug offense & $-$3,473.535$^{***}$ & $-$3,473.535$^{***}$ &  &  &  &  &  &  &  &  \\ 
  & (27.152) & (26.620) &  &  &  &  &  &  &  &  \\ 
  & & & & & & & & & & \\ 
\hline \\[-1.8ex] 
\begin{minipage}{30mm}
	\raggedright
	 Demographic controls
\end{minipage}\vspace{0.5em} & No & Yes & No & Yes & No & Yes & No & Yes & No & Yes \\ 
Observations & 117,216 & 117,216 & 29,304 & 29,304 & 29,304 & 29,304 & 29,304 & 29,304 & 29,304 & 29,304 \\ 
Adjusted R$^{2}$ & 0.198 & 0.229 & 0.040 & 0.109 & 0.110 & 0.166 & 0.073 & 0.148 & 0.076 & 0.139 \\ 
\hline 
\hline \\[-1.8ex] 
\end{tabular} 

    \begin{minipage}{0.96\textwidth}
        \small\vspace{-1.5em}
        \textit{Notes:} All regressions include county and year fixed effects. In ``Misdemeanor rate'' columns, the excluded group is the total/all offenses type. \\
        $^* p<0.1$ \, $^{**} p<0.05$ \, $^{***} p<0.01$
    \end{minipage}
\end{sidewaystable}




%------------------------------------------------------------------------------%

\newpage
%\bibliographystyle{apacite}
%\nocite{*}
%\bibliography{09literature/bibliography.bib}
\printbibliography

\end{document}

%------------------------------------------------------------------------------%

10/28/2025 Notes
Problem: perfect overlap in coal and gas counties production
jobs aren't one-to-one coal to nat gas
there's a transition cost
if outcomes jump up and back down maybe that's the 
no longer addicted cuz now there's a job for me! <--- NO!
safety of coal mines
nationwide gas for robustness

Now we are doing "classic" Bartik

\nb{CONNOR's ASSUMPTIONS ABOVE THIS LINE! ADDIE's ARE BELOW THIS LINE}
\newline
For county $k$ and year $t$, 2SLS specification:

\begin{align}
    \label{eq:struc}
    Y_{kt} &= {D}_{kt}\beta + X_{kt}\gamma + \kappa_k + \tau_t + \varepsilon_{kt} \\ %this is the target model/structural equation, what we would run if we had a reliable measure of "deindustrialization"
    \label{eq:instr}
    Z_{kt} &= \texttt{coal\_share}_{k,2002} \cdot \texttt{coal\_growth}_{t} \\ %this is the Bartik instrument
    \label{eq:1st}
    D_{kt} &= Z_{kt}\phi + X_{kt}\tilde{\gamma} + \tilde{\kappa}_k + \tilde{\tau}_t + \tilde{\varepsilon}_{kt} \\ 
    %stage 1, tildes just denote they may not be the same as in structural model
    \label{eq:2nd}
    Y_{kt} &=  \hat{{D}}_{kt}\beta + X_{kt}\gamma + \kappa_k + \tau_t + \varepsilon_{kt}  %stage 2
\end{align}

Equation \ref{eq:struc} is the structural model where $Y_{kt}$ is a crime outcome, $\gamma_k$ are county fixed effects, $\psi_t$ are time fixed effects, and $X_{kt}$ is a vector of covariates, including \tk{demographics, etc.?}.  $D_{kt}$ is the extent of deindustrialization, which we cannot measure directly and may be endogenous.

Equation \ref{eq:instr} is a \citeauthor{bartik_1991}-style (\citeyear{bartik_1991}) shift-share instrument for deindustrialization that proxies exposure (i.e. vulnerability) to coal firms in county $k$ exiting the market via an interaction between \texttt{coal\_share} and \texttt{coal\_growth}.  It is appropriate to include only one industry in the shift-share formula when the shifts (shocks) happen only in that industry \citep{borusyak_2025}.  In our instrument, coal experiences a negative shock directly in response to the natural gas shock.
\texttt{coal\_share} is the proportion of the county's population employed by a coal firm in 2002, which proxies coal dependence of county $k$.  %Time-independence allows this measure to be exogenous to growth in future years.  
\texttt{coal\_growth} is the national-level growth of coal production, which is location-independent.  Note this measure will be negative in the case of a collapsing industry. 


In order for the instrumented model to be interpreted causally, the following assumptions apply: \\
Beyond basic exogeneity, we must argue that there are properties of the shifts (coal growth rate) and shares (proportion of jobs that are in coal) that make the instrument uncorrelated with the error term (anything outside of ``deindustrialization'' that contributes to crime outcomes) \citep{borusyak_2025}.  Variation in the instrument comes primarily from exogenous shocks to the coal industry via the growth rate.  For example, the rise of natural gas directly coincides with declining coal growth rates.  So now we must argue that the natural gas shock and/or coal share is exogenous to crime outcomes.  Coal share may NOT be exogenous to crime outcomes, there may be differences in the type of people that perform coal jobs that make them different in their propensity to commit crime.  Or, the physical infrastructure in counties with more coal may make it more or less likely that crime is committed.  National coal growth rate 

\tk{what assumption does the following break?} We acknowledge that there is correlation between coal-producing and gas-producing counties, so our instrument does not capture deindustrialization precisely.  However, entry-level coal and natural gas jobs are fundamentally different, requiring different skills and posing different safety risks. \tk{job descriptions}. Additionally, there is a transition period as natural gas firms enter after coal firms begin downsizing and/or exiting \tk{how long?}.  

Assumptions we need to make and defend to identify causality with a Bartik instrument:
\begin{enumerate}
    \item Strict Exogeneity: Shares are exogenous to changes in the error term
    \item Relevance: Testable.
    \item ``The Bartik instrument assumes a pooled exposure research design, where the shares measure differential exposure to common shocks, and identification is based on exogeneity of the shares'' \citep{goldsmith-pinkham_2020}.
\end{enumerate}


11/4 notes from Amanda meeting
