\documentclass{../10extra/creport}

% \usepackage{xcolor} -- xcolor with dvipscolors is imported by the class wait how did you get it red and why does it say TK
% TK is a thing to help search for areas that still need work (CMD+F "TK" will almost always give only the areas where you've flagged "TK"/"to come") -- I have a command \tk defined in the class that makes it red and flags "TK" in the document
%perffffff love it


\title{ECON6590 Final Project: First Deliverable}
\author{Robert Betancourt\\Connor Bulgrin\\Jenny Duan\\Nicholas Skelley\\Adeline Sutton}

\date{October 2025}

\begin{document}
\maketitle

\begin{tcolorbox}
    \begin{itemize}
        \item What is the causal question of interest? What are the $X$ and $Y$ variables? How are they measured, and where are you getting them from?
        \item A brief motivation of why this question is interesting or important.
        \item What research design do you propose to use to answer it?
        \item What data do you propose to use?
        \item Imagine you were just trying to answer this descriptively, not causally:
        \begin{itemize}
            \item Show summary statistics of key variables: main X and Y, plus important controls
            \item Run the bivariate OLS regression of your Y variable on X. Then include a few basic controls.
            \item Create a figure that illustrates this overall association (maybe it’s a binscatter or a bar chart, or something else---will all depend on your question). Show the figure with and without controls (if using, use Cattaneo et al. 2024 for the binscatter with controls)
        \end{itemize}
        \item Describe why these initial analyses may not give a causal answer to the question of interest. How will your proposed research design address this?
    \end{itemize}
\end{tcolorbox}

\setstretch{1.5}

\section{Introduction}
Our broad question of interest is ``How does deindustrialization impact crime outcomes?''  Specifically, we plan to examine the collapse of the coal industry in Appalachia, where coal was a significant industry up until the rise of natural gas in the United States in the early 2000's.  

\section{Design and Data}

\subsection{Data}

\subsubsection{Employment share}

Quarterly Census of Wages and Employment (QCEW), administered by the Bureau of Labor Statistics (BLS).

\subsection{Design and specification}

For county $k$ and year $t$, 2SLS specification:
\begin{align*}
    D_{kt} &= \mu + \phi_1 \; \texttt{coal\_share}_{k,2000} + \phi_2 \; \texttt{gas\_reserves}_{t} + \phi_3 \; \texttt{coal\_share}_{k,2000} \cdot \texttt{gas\_reserves}_{t} + \nu_{kt} \\
    Y_{kt} &= \alpha + \beta \widehat{D}_{kt} + \gamma X_{kt} + \gamma_k + \psi_t + \varepsilon_{kt}
\end{align*}
where $Y_{kt}$ is a crime outcome, $\gamma_k$ are county fixed effects, $\psi_t$ are time fixed effects, and $X_{kt}$ is a vector of covariates, including \tk{demographics, etc.?}.  $D_{kt}$ is extent of deindustrialization. \\
We use a \ref{Bartik_1991)} shift-share instrument for deindustrialization that proxies exposure (i.e. vulnerability) to coal firms in county $k$ exiting the market via an interaction between \textbf{coal\_share} and \textbf{gas\_reserves}.  
\textbf{coal\_share} is the share of the county's population employed by a coal firm in 2000, which proxies coal dependence of county $k$.  Time-independence allows this measure to be exogenous to a gas shock.  
\textbf{gas\_reserves} is nation-wide gas production in year $t$, which proxies the level of the gas shock. The resulting instrument \tk{$Z_{kt}$ is increasing in both of the component variables}.  If a county has no coal jobs, they are not exposed to deindustrialization of the coal industry for any level of gas shock.  Similarly, when there is no gas production, 

\section{Summary Statistics}


\end{document}