\documentclass{10extra/workingpaper}

\usepackage{indentfirst}

\addbibresource{09literature/bibliography.bib}

\title{Criminal Mines: Evaluating the effects of the coal mining industry on incarceration and drug-related crime\\[0.5em]{\Large Methodology}}
\author[*]{Robert Betancourt}
\author[*]{Connor Bulgrin}
\author[$\dagger$]{Jenny Duan}
\author[*]{Nicholas Skelley}
\author[*]{Adeline Sutton}
\affil[*]{\normalsize Department of Economics, Cornell University}
\affil[$\dagger$]{\normalsize Jeb E. Brooks School of Public Policy, Cornell University}
% \keywods
% \jel{K14, L71, O13, O14, O33}

\date{Last updated: \today}

\begin{document}
\maketitle

\setstretch{1.5} % 1.5 spacing

\section{Background}
Between 2001 and 2023, aggregate coal productivity in the US has gone from 1.13 billion short tons to 578 million, a 48.75\% decrease \parencite{coal_data_browser_eia}. In the same period, natural gas growth withdrawals have increased from 24.5 million (million cubic feet) to 45.6 million \parencite{u.s.energyinformation_2025}. Prior research has explored the effect of the coal decline on social effects. \textcite{blonz_2023} find decreased credit scores and increased use of credit, even when accounting for individuals who lost mining-related jobs. \textcite{welch_2020} study the impact of coal activity on local revenue, finding a decrease in local revenue for education per student. Naturally, we expect more poorly funded counties to have worse justice outcomes. A minority of the literature finds the opposite conclusion---for example, \textcite{nason_2025} find some evidence that long-term local reduction in poverty was due to outmigration, wage increases and increased educational investment. We contribute to the literature by studying the relationship between coal deindustrialization and crime at the county level using a difference-in-differences approach that aims to eliminate spillover effects.

Our broad question of interest is ``How does deindustrialization impact crime outcomes?'' More specifically, we test whether a decline in a county's share belonging to coal causes an increase in various crime statistics, including violent, property, and drug-related crime.

\section{Methodology}
We use an event study to estimate the impacts of production decline after a county's largest peak in coal production.  The year immediately follow the peak is the event, so event time is measured in years prior to or after the initial decline. We measure dynamic treatment effects in order to see how coal production decline impacts a county over time.  That is, in each year after the initial decline, \tk{how is that initial decline still impacting a county?}

\textbf{More about continuous treatment event studies}  The following adapts the description from \cite{callaway_2024a} to our purpose.

 
%Potential outcomes and treatment effects
In a standard event study, the treatment is an event; a unit is untreated before the event and treated after.  The potential outcomes are $Y_{kt}(0)$ for individual unit \textit{k}'s outcome in time \textit{t} if \textit{k} was untreated and $Y_{kt}(1)$ if \textit{i} were treated.  We can extend this to the continuous case where $Y_{kt}(d)$ is the outcome of \textit{k} in time \textit{t} had the individual received ``dose'' \textit{d}.  In our context, $Y_{kt}(d)$ would be county \textit{k}'s crime outcome in year \textit{t} if they had an initial decline in coal production from their peak of \tk{slope} \textit{d}.  

While in a standard event study the treatment effect is the difference between the treated and untreated potential outcomes, in the continuous case there are two types of causal effects: the level treatment effect and the causal response.  The level treatment effect of dose \textit{d} is the difference in the potential outcome given the individual received dosage \textit{d} and the potential outcome given the individual was untreated: $Y_{kt}(d)-Y_{kt}(0)$.  The causal response of dose \textit{d} is the slope of the level treatment effect at \textit{d} with respect to the dosage: $Y'_{kt}(d)$; this measures the effect of increasing the dosage incrementally.  In our context, the level treatment effect at \textit{d} would be the difference in the potential outcomes of a county when they had a decline of \textit{d} compared to if they had no decline, while the causal response at \textit{d} would be how the outcome of county \textit{k} would change if they had a decline of $d+\epsilon$ compared to the actual dose \textit{d}, which gets at how impactful having higher rates of decline is.

The two parameters of interest then are the average level treatment effects and the average causal responses.

Our estimating equation is as follows: 

\textbf{Estimating equation}
\[
    y_{kt} = \alpha + \sum_{j\neq -1} \big( \beta_j \cdot \delta_{kt} \cdot \mathbbm{1}[j = t] \big) + \theta_{t} + \phi_{k} + \varepsilon_{kt}
\]

\nb{Using an event study with a continuous treatment based on \citetext{}}
\begin{itemize}
    \item $y_{kt}$ is an outcome in county \textit{k} in calendar year \textit{t}. 
    \item For event time $j$, $j\equiv-1$ in the year of the county $k$'s coal production global (within time span of interest, omitting counties with edge maxima) maximum.
    \item $\delta_{kt}$ is the treatment intensity/dose---the \emph{percent change} in coal production in county $k$ in time $t$ relative to event time $-1$.  This allows us to investigate heterogeneity in treatment effects by how quickly production declines.
    \item $\theta_t$ and $\phi_k$ are calendar year and county fixed effects.
\end{itemize}

The dynamic treatment effects in this event study are $\beta_j$; this is the effect of a county's coal production decline on crime outcomes $j$ periods after the peak of production.

\subsection{Assumptions}
As this is an event study with continuous treatment intensities, it is subject to similar but stronger assumptions in order to interpret the treatment effect as causal (that is, the decline in the coal industry drives changes in crime).  These are:

\begin{enumerate}
\item \textbf{No anticipation} This would imply that agents did not foresee a decline in coal production and as a result changed their criminal behavior.
\item \textbf{Strong parallel trends} \nb{Parallel trends without \emph{strong} PT is still causal, but it identifies ATT and has selection problems; with strong PT, we identify ATE, per \cite{callaway_2024a}}
\end{enumerate}

\begingroup
\itshape\color{purple}
I think strong parallel trends may be supported by observing no pre-trends in the following specification (slightly different from the estimating specification):
\[
    y_{kt} = \eta + \sum_{j \neq -1} \big( \beta_j \cdot \mathbbm{1}[j = t] \big) + \gamma_t \cdot \mathbb{E}[\delta_{kt} \mid t > g] + \theta_t + \phi_k + \varepsilon_{kt}.
\]
\endgroup

\nb{Another robustness check may be to identify a post-peak average slope for coal production (via OLS) and use that slope as the dosage so that the dose doesn't change over time.}

\section{Data}

\begin{itemize}
    \item CJARS columns to keep: 
    \begingroup
    \ttfamily
        fips, cohort\_year, sex, race, age\_group, off\_type, repeat\_contact, fe\_rate, N\_fe\_rate, mi\_rate, N\_mi\_rate, inc\_rate, N\_inc\_rate, par\_rate, N\_par\_rate, pro\_rate, N\_pro\_rate
    \endgroup
\end{itemize}


\section{Results}

\begin{figure}
    \centering
    \includegraphics[width=0.8\linewidth]{06figures/graphs/summary/hist_peak_calyr_all.pdf}
    \begin{minipage}{0.8\linewidth}
        \footnotesize\textit{Notes}: \nb{Includes all counties, including those whose peak production years are the first year they were observed in CJARS.}
    \end{minipage}
\end{figure}

\begin{figure}
    \centering
    \includegraphics[width=0.8\linewidth]{06figures/graphs/summary/hist_peak_calyr_interior.pdf}
    \begin{minipage}{0.8\linewidth}
        \footnotesize\textit{Notes}: \nb{Includes only counties whose peak production years are strictly in the interior of their CJARS observations.}
    \end{minipage}
\end{figure}

\begin{figure}
    \centering
    \includegraphics[width=0.8\linewidth]{06figures/graphs/summary/coal_peak_unweighted.pdf}
    \begin{minipage}{0.8\linewidth}
        \footnotesize\textit{Notes}: \nb{Average county-level $z$-scores---within each county, the coal production quantity in number of SDs from the mean, averaged by year. No weights (see next/weighted fig). Dashed lines 95\% confidence interval.}
    \end{minipage}
\end{figure}

\begin{figure}
    \centering
    \includegraphics[width=0.8\linewidth]{06figures/graphs/summary/coal_peak_prod-wt.pdf}
    \begin{minipage}{0.8\linewidth}
        \footnotesize\textit{Notes}: \nb{Average county-level $z$-scores---within each county, the coal production quantity in number of SDs from the mean, averaged by year. Weighted by the raw quantity of coal produced in the county at event time $-1$. Dashed lines 95\% confidence interval (probably done wrong because I didn't do anything to account for the weights but such is life).}
    \end{minipage}
\end{figure}

\section{Discussion and Plans for Robustness}

state year fixed effects (for robustness because state level policy that responds to recessions)


\newpage
\nocite{callaway_2021, callaway_2025}
\printbibliography

\end{document}